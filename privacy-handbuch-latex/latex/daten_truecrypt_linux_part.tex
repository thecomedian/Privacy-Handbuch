\section{TrueCrypt f�r Linux}
TrueCrypt, haupts�chlich f�r WINDOWS entwickelt, steht auch f�r Linux zum Download zur Verf�gung - eine Option f�r alle, die verschl�sselt Daten zwischen Linux und WINDOWS tauschen m�chten.\\

\textbf{Achtung:} Die Linux-Version verwendet ein eigenes Kernel-Modul f�r den Datenzugriff. Dieses ist nach einem Update des Kernels oder einem Upgrade der Distribition wahrscheinlich unbrauchbar. Die gesamte Installation muss erneut durchlaufen werden. Solange die verwendete Distribution kein zum Kernel passend kompiliertes TrueCrypt bietet, w�rde ich die Verwendung auf den Datenaustausch mit WINDOWS beschr�nken.\\

Mit \textbf{OneKript} steht auf Sourceforge.net ein grafisches GUI f�r Linux zum Download bereit. Dieses GUI ist f�r KDE konzipiert und nutzt den Kommander. \href{http://onekript.sourceforge.net/}{http://onekript.sourceforge.net/}

\subsubsection*{Installation}
Die komplexe Installation ist von en Entwicklern gut gel�st. Man ben�tigt die Quellen des verwendeten Kernel, welche mit den Tools zur Packetverwaltung installiert werden k�nnen.\\

Das TrueCrypt-Archiv ist nach dem Download in das Verzeichnis \textit{/usr/src} zu entpacken. Anschlie�end ist in das Verzeichnis mit den Linux-Sourcen zu wechseln und die �bersetzung mit em Script \textit{build.sh} zu starten. Das Ergebnis wird dann mit \textit{install.sh} installiert.

\begin{verbatim}
  # tar -xzf truecrypt-4.2.a-source-code.tgz -C /usr/src
  # cd /usr/src/truecrypt-4.2.a/Linux
  # ./build.sh
  .....
  # ./install.sh
\end{verbatim}

Die �bersetzung dauerte auf meinem Rechner mehr als 30min. Anscheinend wird dabei auch der Kernel neu kompiliert (aber nicht installiert!).

\subsubsection*{Container �ffnen}
F�r den transparenten Zugriff auf verschl�sselte Container wird der Device-Mapper genutzt. Das �ffnen eines Containers erfolgt in zwei Schritten:
\begin{enumerate}
 \item Als erstes wird der Container dem Device-Mapper unterstellt. Im Beispiel wird die verschl�sselte Containerdatei \textit{geheim.tc} vom USB-Stick als erstes TrueCrypt Device gemappt.
\begin{verbatim}
  > truecrypt -N 1 /media/usbstick/geheim.tc
\end{verbatim}
\item Anschlie�end kann das Device mit dem mount-Befehl in das Dateisystem eingebunden werden, beispielsweise nach \textit{/mnt}.
\begin{verbatim}
   > mount /dev/mapper/truecrypt1 /mnt
\end{verbatim}
\end{enumerate}

Beide Schritte k�nnen mit dem folgenden Kommando zusammengefasst werden, welches automatisch das n�chste freie TrueCrypt Device nutzt.
\begin{verbatim}
  > truecrypt /media/usbstick/geheim.tc /mnt
\end{verbatim}

\subsubsection*{Container schlie�en}
Auch das Schlie�en eines Containers erfolgt in zwei Schritten. Als estes ist das Device mit dem umount-Befehl auszuh�ngen, anschlie�end wird das Mapping des Containers aufgehoben:
\begin{verbatim}
  > umount /mnt
  > truecrypt -d 1
\end{verbatim}

Auch hier gibt es ein Vereinfachung. Das folgende Kommando h�ngte alle TrueCrypt Devices aus und schlie�t die Container:
\begin{verbatim}
  > truecrypt -d
\end{verbatim}

\subsubsection*{Weitere Features}
Es ist auch unter Linux m�glich, versch�sselte Container zu erstellen. Der folgende Befehl startet die Erstellung einer verschl�sselten Containerdatei auf dem USB-Stick. Dabei werden alle Informationen nacheinander abgefragt:
\begin{verbatim}
  > truecrypt -c /media/usbstick/geheim.tc
\end{verbatim}

Da TrueCrypt sehr empfindlich auf Bitfehler im Kopf der Datei reagiert, empfehlen die Entwickler, eine Sicherung des Headers zu speichern um diesen bei Bedarf wieder herstellen zu k�nnen:
\begin{verbatim}
  > truecrypt --backup-header BackupDatei Container
\end{verbatim}

Mit dem folgenden Befehl kann ein gesicherter Header wieder hergestellt werden:
\begin{verbatim}
  > truecrypt --restore-header BackupDatei Container
\end{verbatim}
