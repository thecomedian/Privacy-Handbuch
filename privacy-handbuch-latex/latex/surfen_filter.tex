\section{HTTP-Header filtern}
Neben der Verwendung von Cookies wird auch der Inhalt des HTTP-Header f�r die Gewinnung von Informationen �ber den Surfer genutzt. Das Projekt \textit{Panopticlick} \footnote{ \href{http://panopticlick.eff.org/}{http://panopticlick.eff.org}} der EFF.org zeigt, dass anhand des Fingerprint des HTTP-Headers 80\% der Surfer eindeutig erkennbar sind. Eine Verkn�pfung dieser Information �ber mehrere Websites hinweg kann eine Verfolgung von Nutzern erm�glichen. Kombiniert man diese Verfolgung mit Daten von Sozialen Netzen (Facebook, Xing), ist eine vollst�ndige Deanonymiserung m�glich.
\begin{itemize}
\item Beispiel \textbf{User-Agent}: Die meisten Browser senden Informationen �ber den verwendeten Browser und das Betriebssystem. Ein Beispiel zeigt, wie detailliert der Browser Auskunft gibt:
\begin{verbatim}
Mozilla/5.0 (Macintosh; U; PPC Mac OS X; de-DE) AppleWebKit/419.3 
(KHTML, like Gecko) Safari/419.3
\end{verbatim}
Beim US-Reiseportal Orbitz werden Surfern mit MacOS (am User-Agent erkennbar) die Hotelzimmer 20-30 Dollar teuerer angeboten, als anderen Kundern\footnote{ \href{http://heise.de/-1626368}{http://heise.de/-1626368}}. Au�erdem k�nnen anhand der Informationen gezielt L�cken in der verwendeten Software ausgenutzt werden.\\


\item Erg�nzende Informationen wie zum Beispiel die bevorzugte \textbf{Sprache}, installierte \textbf{Schrift�arten} und \textbf{Gr��e des Browserfensters} k�nnen einen individuellen Fingerprint des Browsers ergeben. Viele Werte k�nnen per Javascript ausgelesen werden Bei der Google-Suche und beim Trackingdienst Multicounter\footnote{ \href{http://www.multicounter.de/features.html}{http://www.multicounter.de/features.html}} wird die innere Gr��e des Browser�fensters ausgelesen. Die Firma bluecave\footnote{ \href{http://www.bluecava.com/}{http://www.bluecava.com}} nutzt z.B. im Trackingscript \textit{BCAL5.js} u.a. Informationen �ber installierte Schriftarten.\\

Deshalb sollte man Javascript nur f�r vertrauensw�rdige Webseiten erlauben und das Auslesen der Werte behindern (soweit m�glich).
\end{itemize} 

