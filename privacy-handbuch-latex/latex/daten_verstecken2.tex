\section{stegdetect}
Auch die Gegenseite ist nicht wehrlos. Manipulationen von steghide, F5, outguess, jphide usw. k�nnen z.B. mit \textit{stegdetect} \footnote{ \href{http://www.outguess.org/download.php}{http://www.outguess.org/download.php}} erkannt werden. Ein GUI steht mit \textit{xsteg} zur Verf�gung, die Verschl�sselung der Nutzdaten kann mit \textit{stegbreak} angegriffen werden. Beide Zusatzprogramme sind im Paket enthalten.\\

Der Name \textit{stegdetect} ist eine Kurzform von \textit{Steganografie Erkennung}. Das Programm ist nicht nur f�r den Nachweis der Nutzung von \textit{steghide} geeignet, sondern erkennt anhand statistischer Analysen auch andere Tools.

Auch \textit{stegdetect} ist ein Tool f�r die Kommandozeile. Neben der zu untersuchenden Datei kann mit einem Parameter -s die Sensitivit�t eingestellt werden. Standardm��ig arbeitet stegdetect mit einer Empfindlichkeit von 1.0 ziemlich oberfl�chlich. Sinnvolle Werte liegen bei 2.0...5.0.
\begin{verbatim}
   > stegdetect -s 2.0 bild.jpg
   F5(***)
\end{verbatim}
Im Beispiel wird eine steganografische Manipulation erkannt und vermutet, dass diese mit dem dem Tool F5 eingebracht wurde (was nicht ganz richtig ist da \textit{steghide} verwendet wurde).\\

\textbf{Frage:} Was kann man tun, wenn auf der Festplatte eines mutma�lichen Terroristen 40.000 Bilder rumliegen? Muss man jedes Bild einzeln pr�fen?\\

\textbf{Antwort:} Ja - und das geht so:

\begin{enumerate}
 \item Der professionelle Forensiker erstellt zuerst eine 1:1-Kopie der zu untersuchenden Festplatte und speichert das Image z.B. in \textit{terroristen\_hda.img}
\item Mit einem kurzen Dreizeiler scannt er alle 40.000 Bilder in dem Image:
\begin{verbatim}
> losetup -o $((63*512)) /dev/loop0 terroristen_hda.img
> mount -o ro,noatime,noexec /dev/loop0 /mnt
> find /mnt -iname "*.jpg" -print0 | xargs -0 stegdetect -s 2.0 >> ergebnis.txt
\end{verbatim} 
(F�r Computer-Laien und WINDOWS-Nutzer sieht das vielleicht nach Voodoo aus, f�r einen Forensiker sind das jedoch Standardtools, deren Nutzung er aus dem �rmel sch�ttelt.)
\item Nach einiger Zeit wirft man eine Blick in die Datei \textit{ergebnis.txt} und wei�, ob es etwas interessantes auf der Festplatte des Terroristen gibt.
\end{enumerate}
