\item[Ubuntu:] F�r Ubuntu kann man das offizielle PPA Repository der I2P Maintainer nutzen. Dieses Repository enth�lt nur den I2P-Router. Es wird mit folgenden Kommandos aktiviert und danach der I2P-Router installiert: 
\begin{verbatim}
  > sudo apt-add-repository ppa:i2p-maintainers/i2p
  > sudo apt-get update
  > sudo aptitude install i2p 
\end{verbatim}
Au�erdem gibt es das I2P PPA-Repository von KYTV. Dieses Repository enth�lt neben dem I2P-Router weitere n�tzliche Software wie I2P-Bote. Das Repository wird mit folgendem Kommando aktiviert: 
\begin{verbatim}
  > sudo apt-add-repository ppa:i2p.packages/i2p
\end{verbatim}
Danach kann man wie �blich alles n�tige auf die Platte sp�len: 
\begin{verbatim}
  > sudo apt-get update
  > sudo aptitude install i2p i2p-bote
\end{verbatim}

\item[Linux:]  Installieren Sie als erstes Java (Paket: \textit{default-jre}) mit der Paketverwaltung der Distribution. Anschlie�end kann der I2P-Router installiert werden. Den Installer \textit{i2pinstall-0.x.y.jar} findet man auf der Downloadseite des Projektes\footnote{ \href{http://www.i2p2.de/download.html}{http://www.i2p2.de/download.html}}. Nach dem Downlad startet man den Installer und w�hlt die Sprache sowie Verzeichnis f�r die Installation:
\begin{verbatim}
 > java -jar i2pinstall-*.jar
\end{verbatim}
In dem neu angelegten Installationsverzeichnis findet man das Script zum Starten/Stoppen des I2P-Routers:
\begin{verbatim}
 > ~/i2p/i2prouter start
\end{verbatim}

Stoppen l�sst sich der Router in der Router-Konsole im Webbrowser unter  \href{http://localhost:7657}{http://localhost:7657} mit Klick auf den Link \textit{shutdown} oder obiges Kommando mit der Option \textit{stop}.

\item[Linux (advanced):] K. Raven hat eine umfassende Anleitung geschrieben, wie man den I2P-Router in einer chroot-Umgebung installiert und mit AppAmor zus�tzlich absichert. Lesenswert f�r alle, die es richtig gut machen wollen. Link: \href{http://wiki.kairaven.de/open/anon/chrooti2p}{http://wiki.kairaven.de/open/anon/chrooti2p}