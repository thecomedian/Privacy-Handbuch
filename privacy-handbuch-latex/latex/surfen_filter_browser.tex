\subsection{Mozilla Firefox konfigurieren}
Auch Mozilla Firefox kann einen installierten Content-Filter nutzen.\\

\begin{figure}[htb]
\begin{center}
\includegraphics[scale=0.60]{../screenshots/browser_20.png}
\caption{Einstellungsdialog in Firefox 2.0}
\label{abb:firefox_einst_verb_20}
\end{center}
\end{figure}

Im Browser \textbf{Firefox 1.5} ist hierf�r der Dialog \textit{Einstellungen} zu �ffnen und zur Sektion \textit{Allgemein} zu wechseln. Die Konfiguration des genutzten Proxy erreicht man �ber den Button \textit{Verbindungs-Einstellungen}.\\

Im Browser \textbf{Firefox 2.0} ist der Dialog \textit{Einstellungen} �berarbeitet worden. Die Verbindung zum Internet kann in der Sektion \textit{Erweitert} auf dem Reiter \textit{Netzwerk} konfiguriert werden. Hier findet man den Button \textit{Einstellungen}, der ebenfalls den in Bild \ref{abb:firefox_einst_verb} gezeigte Dialog �ffnet.\\

\begin{figure}[htb]
\begin{center}
\includegraphics[scale=0.60]{../screenshots/browser_2.png}
\caption{Verbindungs-Einstellungen}
\label{abb:firefox_einst_verb}
\end{center}
\end{figure}

Es ist die Option \textit{manuelle Proxy-Konfiguration} zu aktivieren, als Host der eigene Rechner mit \textit{localhost} anzugeben und der Port einzutragen, an welchem der Content-Filter lauscht. F�r Proxomitron ist Port \textit{8080} anzugeben, f�r Privoxy Port \textit{8118}.\\

Au�erdem kann eine Liste von Ausnahmen angegeben werden, eine Liste von Websites, welche ohne Content-Filter besucht werden sollen. Hier sind alle vertrauensw�rdigen Websites einzutragen, welchen mit eingeschaltetem Content-Filter nicht korrekt funktionieren. In jedem Fall sollten lokale Webseiten nicht gefiltert werden (\textit{localhost, 127.0.0.1}).\\

L�uft der Content-Filter nicht auf dem eigenen Rechner sondern auf einem Gateway, ist statt \textit{localhost} der Name oder die IP-Adresse dieses Rechners einzutragen.
 