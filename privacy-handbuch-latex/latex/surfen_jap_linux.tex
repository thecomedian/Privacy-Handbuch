\item F�r \textbf{Ubuntu} sowie \textbf{Debian} bietet JonDos fertige Pakete. Um das Software Repository der JonDos GmbH zu nutzen, ist in der Datei \textit{/etc/apt/sources.list} folgende Zeile einzuf�gen und DISTRI durch die verwendete Distribution zu ersetzen (\textit{squeeze, wheezy, sid, maverick, natty, oneiric} oder\textit{ precise}):
\begin{verbatim}
 deb http://debian.anonymous-proxy-servers.net DISTRI main
\end{verbatim} 
Das Repository ist mit dem OpenPGP-Key \textit{0xF1305880} signiert, der unter folgender Adresse zum Download bereit liegt:\\ \href{https://anonymous-proxy-servers.net/downloads/JonDos_GmbH.asc}{https://anonymous-proxy-servers.net/downloads/JonDos\_GmbH.asc}\\

Nach dem Download ist der Schl�ssel in den APT-Keyring einzuf�gen:
\begin{verbatim}
 sudo apt-key add JonDos_GmbH.asc
\end{verbatim} 
Danach kann das Paket \textit{jondo} wie �blich installiert werden.
\begin{verbatim}
   > sudo apt-get update 
   > sudo aptitude install jondo jondofox-de
\end{verbatim}
Nach der Installation kann man JonDo �ber das Programmmen� starten \textit{Applications -> Internet -> Jondo} oder auf der Kommandozeile mit \textit{jondo}. Wenn man das Bowserprofil \textit{JonDoFox} f�r Firefox/Iceweasel gleich mit installiert, findet man auch einen fertig konfigurierten Browser in der Men�gruppe \textit{Internet}.

\item F�r andere \textbf{Linux/UNIX} Versionen ist als erstes ein Java Runtime Environment zu installieren. Aktuelle Linux-Distributionen bieten die Pakete \textit{openjdk-6-jre} oder \textit{openjdk-7-jre}, die mit dem bevorzugten Paketmanager installiert werden k�nnen.\\

Hinweis: Im Gegensatz zu Windwos wird dabei kein Java Plug-in f�r die Browser installiert. Man braucht nicht verzweifelt danach zu suchen um es zu deaktivieren.\\

Anschlie�end l�dt das Archiv jondo\_linux.tar.bz2 von der Downloadseite herunter, entpackt es und installiert JonDo mit folgendem Kommando:
\begin{verbatim}
   > tar -xjf jondo_linux.tar.bz2
   > cd jondo_linux
   > sudo ./install_jondo
\end{verbatim}
Die Installationsroutine richtet Men�eintr�ge in der Programmgruppe Internet f�r die g�ngigen Desktop Umgebungen ein. Auf der Kommandozeile startet man das Proxyprogramm mit \textit{jondo}.\\

Deinstallieren kann man das Programm mit:
\begin{verbatim}
   > sudo jondo --remove
\end{verbatim}

