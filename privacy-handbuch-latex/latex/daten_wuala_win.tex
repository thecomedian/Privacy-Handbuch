\subsubsection{Online-Speicher mit Wuala}
Wuala ist ein Cloud-Speicher der 5 GB Speicherplatz kostenlos anbietet, f�r mehr muss man bezahlen. Bezahlung f�r zus�tzlichen Speicher ist nur via PayPal.com auf der Webseite \href{https://www.wuala.com}{https://www.wuala.com} m�glich.\\

Man kann den Datenspeicher als Backup-Medium nutzen, Daten auf mehreren Rechnern synchronisieren oder im Team verteilen. Die Client Software verschl�sselt die Daten auf dem eigenen Rechner bevor sie in den Online-Speicher �bertragen werden. Die Software gibt es f�r Windows, Linux und MacOS sowie f�r Android ind iPhone.\\

Die \textbf{Installation} ist einfach. F�r Windows steht auf der Webseite ein Setup-Programm zum Downlaod bereit. Nach dem Download ist das Programm zu starten und den Anweisungen des Assistenten zu folgen.\\

Beim ersten Start muss man einen Account anlegen. Das Passwort ist sehr wichtig! Es gibt keine M�glichkeit, an die Daten im Online-Speicher zu kommen, wenn man das Passwort vergessen hat. Es gibt auch keine M�glichkeit zum Passwort-Reset!  Die E-Mail Adresse ist unwichtig, es gibt nur eine Verwendung. Wenn man das Passwort vergessen hat, wird die Passwort-Merkhilfe an diese E-Mail Adresse gesendet. Wegwerf-Adressen werden akzeptiert.\\

\begin{figure}[htb]
\begin{center}
\includegraphics[scale=0.45]{../screenshots/wuala-config.png}
\caption{Wuala Konfiguration}
\label{abb:wualaconfig}
\end{center}
\end{figure}

In der Konfiguration kann man die Up- und Download Geschwindigkeit an den eigenen Internetzugang anpassen.\\

Mit einem HTTP-Proxy kann man die Verbindung zum Cloud-Speicher anonymisieren. JonDonym kann out-of-the-box als Anonymisierungsdienst verwendet werden. Bei der Nutzung von Premium-Diensten gibt es kaum Geschwindigkeitseinbu�en. Tor bietet nur einen SOCKS Proxy und kann deshalb nicht direkt verwendet werden. Man ben�tigt zus�tzlich eine HTTP-Proxy (Polipo oder Privoxy), die richtig konfiguriert den Datenverkehr an Tor weiterleiten k�nnen.\\

Der Wuala-Client stellt unter Windows ein zus�tzliches Laufwerk \textit{Wuala} zur Verf�gung. Alle Daten, die man in dieses Laufwerk kopiert, werden in den Online-Speicher geladen. Au�erdem stehen die Daten aus dem Online-Speicher in diesen Verzeichnissen zum wahlfreien Zugriff zur Verf�gung.\\

Im Wuala-GUI kann man au�erdem Backups hinzuf�gen, Verzeichnisse synchronisieren oder Daten f�r Gruppen freigeben. F�r die ersten beiden Funktionen kann ein beliebiger Ordner auf dem lokalen Rechner mit einem Ordner im Wuala-Drive verbunden werden. Bei einem Backup gehen die Daten nur vom eigenen Rechner in den Online-Speicher. Bei einer Synchronisation werden auch die Daten auf dem eigenen Rechner modifiziert, wenn sich Daten im Online-Speicher �ndern. Diese Funktion eignet sich, wenn Daten auf mehreren Rechnern identisch sein sollen.\\

Mit privaten oder �ffentlichen Gruppen kann man den Inhalt eines Ordners im Wuala-Speicher mit anderen Nutzern teilen. �ffentliche Ordner k�nnen auch im Internet zug�nglich gemacht werden. Unter \textit{https://www.wuala.com/PrivacyHandbuch} ist beispielsweise der LaTEX Quelltext des Privacy-Handbuches verf�gbar. Dabei sollte man darauf achten, die Schreibrechte in der Gruppe restriktiv zu setzen, damit nicht irgendwelche Vandalen ihren M�ll dort abladen. 

\begin{figure}[htb]
\begin{center}
\includegraphics[scale=0.5]{../screenshots/wuala-drive.png}
\caption{Wuala-Laufwerk im Windows Explorer}
\label{abb:wualadrive}
\end{center}
\end{figure}