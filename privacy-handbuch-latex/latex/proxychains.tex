\section{Nicht-proxyf�hige Internetanwendungen}
Mit Hilfe von proxifier-Tools k�nnen auch Applicationen anonymisoert werden, die keine Unterst�tzung f�r Proxies bieten. Unter Windows kann man Widecap nutzen (\href{http://widecap.com}{http://widecap.com}). Linux/UNIX Distributionen enthalten das nette Tool \textit{proxychains}.

\subsubsection*{proxychains f�r Linux/UNIX}
Nachdem man \textit{proxychains} mit dem Paketmanager der Distribution installiert hat, ist die Standardkonfiguration bereits f�r Tor Onion Router vorbereitet. Sollen statt Tor die Premium-Dienste von  JonDonym genutzt werden, ist eine Konfigurationsdatei in \textit{\$(HOME)/.proxychains/proxychains.conf} zu erstellen:
\begin{verbatim}
 strict_chain
 proxy_dns 
 [ProxyList]
 http 127.0.0.1 4001 
\end{verbatim}

Um den Traffic beliebiger Anwendungen zu anonymisieren, startet man die Anwendung unter Kontrolle von \textit{proxychains}. Unbeobachtete Administration eines Servers mittels SSH ist m�glich mit: 
\begin{verbatim}
 proxychains ssh user@server.tld
\end{verbatim}

Den Instant Messenger Kopete (KDE/Linux) kann man anonymisieren: 
\begin{verbatim}
 proxychains kopete
\end{verbatim}