
\subsection{Tor Hidden Services}
Das Tor Netzwerk erm�glicht nicht nur den anonymen Zugriff auf herk�mmliche Angebote im Web sondern auch die Bereitstellung anonymer, zensurresitenter und schwer lokalisierbarer Angebote auf den Tor-Nodes. Der Zugriff auf die Tor Hidden Services ist nur �ber das Tor Netzwerk m�glich. Eine kryptische Adresse mit der Top-Level Domain .onion dient gleichzeitig als Hashwert f�r ein System von Schl�sseln, welches sicherstellt, dass der Nutzer auch wirklich mit dem gew�nschten Dienst verbunden wird. Die vollst�ndige Anonymisierung des Datenverkehrs stellt sicher, dass auch die Betreiber der Angebote nur sehr schwer ermittelt werden k�nnen.\\

Es gibt mehere Angebote im normalen Web, die zus�tzlich als Tor Hidden Service anonym und unbeobachtet erreichbar sind. 
\begin{itemize}
  \item Riseup.net bietet Kommunikationsdienste f�r politische Aktivisten. Alle Dienste wie E-Mail (POP3, SMTP, IMAP), Jabber XMPP usw. gibt es als Riseup's Hidden Services.
 \item Die Suchmaschine DuckDuckGo ist unter der Adresse \href{http://3g2upl4pq6kufc4m.onion}{http://3g2upl4pq6kufc4m.onion} zu finden. F�r Firefox gibt es bei Mycroft ein Add-on f�r die Suchleiste, das diesen Hidden service nutzt.
 \item Wikileaks gibt es unter \href{http://isax7s5yooqgelbr.onion}{http://isax7s5yooqgelbr.onion}.
 \item https://keys.indymedia.org ist ein Webinterface f�r die Suche nach OpenPGP-Schl�sseln. Es ist ein Hidden Service erreichbar unter \href{http://qtt2yl5jocgrk7nu.onion}{http://qtt2yl5jocgrk7nu.onion}.
 \item awxcnx.de gibt es auch unter \href{http://a5ec6f6zcxtudtch.onion}{http://a5ec6f6zcxtudtch.onion} 
\end{itemize}

Meine ``Sammlung`` an reinen Tor Hidden Services enth�lt im Moment: 
\begin{itemize}
 \item 34x Angebote, die kinderpornografischen Schmutz zum Download anbieten (ausschlie�lich und teilweise zus�tzlich zu anderen Inhalten).
 \item 3x Angebote zum Thema \textit{Rent a Killer}. Ein Auftragsmord kostet offenbar nur 20.000 Dollar (wenn diese Angebote echt sind).
 \item Ein Angebot f�r gefakete Ausweisdokumente (aufgrund der mit Photoshop o.�. bearbeiteten Screenshots der Beispieldokumente auf der Webseite halte ich das Angebot selbst f�r einen Fake).
 \item Mehrere Handelsplattformen f�r Drogen.
 \item Einige g�hnend langweilige Foren \& Blogs mit 2-3 Beitr�gen pro Monat.
 \item Einige Index-Seiten mit Listen f�r verf�gbare Hidden Services
wie das legend�re \textit{HiddenWiki} oder das neuere \textit{TorDirectory}. In diesen Index Listen findet man massenweise Verweise auf Angebote mit Bezeichnungen wie \textit{TorPedo}, \textit{PedoVideoUpload}, \textit{PedoImages}. Nach Beobachtung von ANONYMOUS sollen 70\% der Besucher des \textit{HiddenWiki} die Adult Section aufsuchen, wo dieses Schmutzzeug verlinkt ist. 
\end{itemize}

Es gibt also kaum etwas, dass ich weiterempfehlen m�chte.\\

Vielleicht kann man f�r unbeobachtete und vorratsdatenfreie Kommunikation die folgenden Dienste nennen:
\begin{itemize}
 \item \textbf{TorMail} unter der Adresse \href{http://jhiwjjlqpyawmpjx.onion}{http://jhiwjjlqpyawmpjx.onion} bietet SMTP und POP3. Es k�nnen auch E-Mails aus dem normalen Web unter \textit{xxx@tormail.net} empfangen werden.
 \item \textbf{TorPM} unter \href{http://4eiruntyxxbgfv7o.onion/pm/}{http://4eiruntyxxbgfv7o.onion/pm/} bietet die M�glichkeit, Textnachrichten ohne Attachments unbeobachtet auszutauschen. Der Dienst erfordert das Anlegen eines Accounts. Das Schreiben und Lesen der Nachrichten erfolgt im Webinterface. 
 \item \textbf{SimplePM:} \href{http://4v6veu7nsxklglnu.onion/SimplePM.php}{http://v6veu7nsxklglnu.onion/SimplePM.php} arbeitet komplett ohne Anmeldung. Beim Aufruf erh�lt man zwei Links: einen Link kann man als Kontakt-Adresse versenden, den zweiten Link f�r die InBox sollte man als Lesezeichen speichern. Es k�nnen einfache Textnachrichten via Webinterface geschrieben und gelesen werden.
 \item \textbf{OpenPGP Keyserver:} \href{http://qtt2yl5jocgrk7nu.onion/}{http://qtt2yl5jocgrk7nu.onion} ist ein Webinterface f�r die Suche nach OpenPGP-Schl�sseln. Es ist ein Hidden Service f�r https://keys.indymedia.org.

 \item \textbf{Jabber-Server} f�r Instant-Messaging via XMPP:
 \begin{itemize}
  \item ch4an3siqc436soc.onion:5222
  \item ww7pd547vjnlhdmg.onion:5222
  \item 3khgsei3bkgqvmqw.onion:5222
 \end{itemize}
\item \textbf{Jabber-Server}
\begin{itemize}
\item p4fsi4ockecnea7l.onion:6667 (Tor Hidden Service des Freenode Netzwerk, kann nur mit registrierten Nicks genutzt werden.)
 \end{itemize}
\end{itemize}

F�r die Tor Hidden Services gibt es kein Vertrauens- oder Reputationsmodell. Es ist unbekannt, wer die Hidden Services betreibt und es ist damit sehr einfach, einen Honeypot aufzusetzen. Anonym bereitgestellten Dateien sollte man immer ein gesundes Misstrauen entgegen bringen und in Diskussionen wird aus dem Deckmantel der Anonymit�t heraus alles m�gliche behauptet.
