\subsection{GnuPG f�r WINDOWS}

Diese simple Verschl�sselung klappt allerdings unter WINDOWS nicht auf Anhieb. Es ist zuerst die n�tige Software zu installieren. Folgende Varianten kann man probieren:
\begin{enumerate}
 \item Das Programmpaket \textbf{gpg4win} enth�lt eine Erweiterung f�r den Windows Explorer, die zus�tzliche Men�punkte im Kontextmen� einer Datei bzw. Verzeichnisses einf�gt.\\

 Download: \href{http://www.gpg4win.org/}{http://www.gpg4win.org}

 \item Das Programmpaket \textbf{GpgSX} enth�lt neben einer aktuellen Version von GnuPG auch einige grafische Tools, welche die Arbeit vereinfachen. Neben einer Schl�sselverwaltung wird auch eine Erweiterung f�r den Explorer installiert, die Verschl�sseln und Entschl�sseln von Dateien mit wenigen Mausklicks erm�glicht.\\

 Download: \href{http://gpgsx.berlios.de/}{http://gpgsx.berlios.de/}

 \item F�r Nutzer, die es gern etwas einfacher und �bersichtlicher m�gen, gibt es die Tools \textbf{gpg4usb} \href{http://gpg4usb.cpunk.de/}{http://gpg4usb.cpunk.de} oder \textbf{Portable PGP} \href{http://ppgp.sourceforge.net/}{http://ppgp.sourceforge.net} (eine Java-App). Diese kleinen Tools k�nnen Texte und Dateien ver- bzw. entschl�sseln und sind auch USB-tauglich. Sie k�nnen auf einem USB-Stick f�r mitgenommen werden. Sie speichern die OpenPGP-Keys auf dem Stick und integrieren sich nicht in den Explorer.

\end{enumerate}

