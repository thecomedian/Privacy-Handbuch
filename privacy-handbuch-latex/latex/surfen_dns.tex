\section{Unzensierte DNS-Server nutzen}
\includegraphics[scale=0.6]{../grafiken/zensursula_seite_04.png}\\
Am 17.04.09 unterzeichneten diese Provider einen geheimen Vertrag mit dem BKA, in welchem sie sich verpflichtenten, den Zugriff auf eine vom BKA bereitsgestellte Liste von Websites zu sperren. Soweit bekannt wurde, soll die Sperrung haupts�chlich durch Kompromittierung des DNS-Systems erfolgen.\\

\textbf{Hinweis:} Diese leicht zu umgehende Sperre ist im internationalen Vergleich die Ausnahme. Lediglich Australien hat einen vergleichbaren Weg gew�hlt. Die folgenden Hinweise zur Umgehung der Zensur durch Nutzung unzensierter DNS-Server k�nnen nicht auf andere L�nder mit technisch hochger�steter Zensur-Infrastruktur �bertragen werden.\\

Bevor man als Kunde dieser Provider ernsthaft �ber die Nutzung alternativer DNS-Server nachdenkt, sollte man die M�glichkeit eines \textbf{Provider-Wechsels} pr�fen. Das hat folgende Vorteile:
\begin{enumerate}
 \item Man unterst�tzt Provider, die sich gegen die Einschr�nkung der Grundrechte wehren, und �bt Druck auf die Zensur-Provider aus.
 \item Es ist auf f�r IT-Laien eine sichere L�sung, unzensierte DNS-Server zu nutzen, da m�glicherweise Zensur-Provider den Datenverkehr auf eigene, zensierte DNS-Server umlenken, ohne dass man es als Nutzer bemerkt. So leitet Vodafone bspw. bereits seit Juli 09 im UMTS-Netz DNS-Anfragen auf die eigenen Server um. Im DFN Forschungsnetz soll die Nutzung unzensierter DNS-Server durch Sperrung des Port 53 unterbunden werden.
\end{enumerate}

Die deutschen Provider Manitu (\href{http://www.manitu.de}{http://www.manitu.de}) und SNAFU (\href{http://www.snafu.de}{http://www.snafu.de}) lehnten die Sperren ab und werden sie auch nicht umsetzen. SNAFU bietet seinen Kunden an, via Webinterface alternative, unzensierte DNS-Server f�r den eigenen Account zu konfigurieren. Damit entfallen die im folgenden beschrieben Spielereien am privaten Rechner und man hat mit Sicherheit einen unzensierten Zugang zum Web.

\subsubsection*{Was ist ein DNS-Server}
\begin{enumerate}
 \item Der Surfer gibt den Namen einer Website in der Adressleiste des Browsers ein. (z.B. \textit{https://www.awxcnx.de})
\item Daraufhin fragt der Browser bei einem DNS-Server nach der IP-Adresse des Webservers, der die gew�nschte Seite liefern kann.
\item Der DNS-Server sendet eine Antwort, wenn er einen passenden Eintrag findet. (z.B. \textit{62.75.219.7}) oder NIXDOMAIN, wenn man sich vertippt hat.
\item Dann sendet der Browser seine Anfrage an den entsprechenden Webserver und erh�lt als Antwort die gew�nschte Website.
\end{enumerate}
Ein kompromittierter DNS-Server sendet bei Anfrage nach einer indexierten Website nicht die korrekte IP-Adresse des Webservers an den Browser, sondern eine manipulierte IP-Adresse, welche den Surfer zu einer Stop-Seite f�hren soll.\\

Die Anzeige der Stop-Seite bietet die M�glichkeit, die IP-Adresse des Surfers zusammen mit der gew�nschten, aber nicht angezeigten Webseite zu loggen. Mit den Daten der Vorratsdatenspeicherung k�nnte diese Information personalisiert werden.\\

(Diese Darstellung ist sehr vereinfacht, sie soll nur das Prinzip zeigen. Praktische Versuche, das DNS-System zu manipulieren, haben meist zu komplexen Problemen gef�hrt.)

\subsubsection*{Nicht-kompromittierte DNS-Server}

Statt der kompromittierten DNS-Server der Provider kann man sehr einfach unzensierte Server nutzen.  Einige DNS-Server k�nnen auch auf Port 110 (TCP-Porotoll) angefragt werden, falls einige Provider den DNS-Traffic auf Port 53 zum eigenen Server umleiten oder behindern. Wir gehen bei der Konfiguration f�r Windows und Linus darauf n�her ein.\\

Die Swiss Privacy Foundation stellt folgende unzensierten DNS-Server mit aktiviertem DNSSEC zur Verf�gung:
\begin{verbatim}
   87.118.85.241     (DNS-Ports: 53, 110)
   77.109.138.45     (DNS-Ports: 53, 110)
   77.109.139.29     (DNS-Ports: 53, 110)
\end{verbatim}

Der FoeBud bietet einen unzensierten DNS-Server:
\begin{verbatim}
   85.214.20.141
\end{verbatim}

Und der CCC hat nat�rlich auch einen Unzensierten, aber ohne DNSSEC:
\begin{verbatim}
   213.73.91.35
\end{verbatim}

