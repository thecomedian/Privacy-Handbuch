\section{JavaScript}
JavaScript ist eine der Kerntechniken des modernen Internet, birgt aber auch einige Sicherheitsrisiken.
\begin{enumerate}
\item Mit Hilfe von Javascript kann man ein Vielzahl von Informationen �ber den Browser und das Betriebssystem auslesen. Bildschirmgr��e, Farbeinstellungen, installierte Plugins und Hilfs-Applikationen.... Die Website \href{http://browserspy.dk}{http://browserspy.dk} zeigt eine umfangreiche Liste.\\

Diese Informationen k�nnen zu einem individuellen Fingerabdruck verrechnet werden. Anhand dieses Fingerabdruck kann der Surfer wiedererkannt werden, auch wenn er die IP-Adresse mit VPNs oder Anonymisierungsdiensten verschleiert. Die EFF geht davon aus, dass diese Methode von vielen Datensammlern genutzt wird.
\begin{itemize}
 \item \textit{Yahoo! Web Analytics} nutzt Javascript Tracking Code, wenn Cookies blockiert werden.
 \begin{quote}
  \textit{In case Yahoo! Web Analytics cannot set a cookie, the system can still retrieve information from the JavaScript tracking code, the IP address and the web browser user agent. \footnote{ \href{http://help.yahoo.com/l/us/yahoo/ywa/documentation/install\_guide/ig\_get\_started.html}{http://help.yahoo.com/l/us/yahoo/ywa/documentation/install\_guide/ig\_get\_started.html}} }
 \end{quote}  
 \item Ein weiteres Beispiel ist die Firma \textit{bluecave} \footnote{ \href{http://www.bluecava.com}{http://www.bluecava.com}}. Das Trackingscript \textit{BCAL5.js} sammelt Informationen zur verwendeten Software, installierte Schriftarten, Bildschirmgr��e, Browser Plug-ins und ein paar mehr Daten, um daraus einen individuellen Fingerprint zu berechnen. \textit{bluecave} protzt damit, 99\% der Surfer zu erkennen.
 \item Der Trackingdienst Multicounter \footnote{ \href{http://www.multicounter.de/features.html}{http://www.multicounter.de/features.html}} und die Google Suche speichern die per Javascript ausgelesene Bild�schirm�gr��e als individuelles Merkmal.
\end{itemize}

\item Einige EverCookie Techniken nutzen Javascript, um zus�tzliche Markierungen im Browser zu hinterlegen und gel�schte Tracking Cookies wiederherzustellen.

\item Durch Einschleusen von Schadcode k�nnen Sicherheitsl�cken ausgenutzt und der der Rechner kann kompromittiert werden. Das Einschleusen von Schadcode erfolgt dabei auch �ber vertrauensw�rdige Webseiten, beispielsweise mit Cross Site Scripting, wenn diese Websites nachl�ssig programmiert wurden. Werbebanner k�nnen ebenfalls b�s�artigen Javascriptcode transportieren. Im Januar 2013 lieferten die Server des Werbe�netzwerkes OpenX Scripte aus, die Rechner durch Ausnutzung mehrerer Sicherheitsl�cken im Internet Explorer kompromittierten.\footnote{ \href{http://heise.de/-1787511}{http://heise.de/-1787511}}
\end{enumerate}

Ein generelles Abschalten ist heutzutage nicht sinnvoll. �hnlich dem Cookie-Management ben�tigt man ein Whitelisting, welches JavaScript f�r vertrauensw�rdige Websites zur Erreichung der vollen Funktionalit�t erlaubt, im allgemeinen jedoch deaktiviert. Gute Webdesigner weisen den Nutzer darauf hin, dass ohne Javascript eine deutliche Einschr�nkung der Funktionalit�t zu erwarten ist. 


