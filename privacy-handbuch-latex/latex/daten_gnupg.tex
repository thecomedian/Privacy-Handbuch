\section{Quick and Dirty mit GnuPG}
Eine M�glichkeit ist die Verschl�sselung einzelner Dateien mit GnuPG oder PGP. Einfach im bevorzugten Dateimanager mit der rechten Maustaste auf eine Datei klicken und den Men�punkt \textit{Datei verschl�sseln} w�hlen. Mit der Auswahl eines Schl�ssels legt man fest, wer die Datei wieder entschl�sseln kann. F�r Backups wird in der Regel der eigene Schl�ssel verwendet. Es ist auch m�glich, mehrere Schl�ssel f�r verschiedene Empf�nger zu nutzen. Die Verwaltung der OpenPGP Schl�ssel ist im Kapitel \textit{E-Mails verschl�sseln} beschrieben. Anschlie�end ist das unverschl�sselte Orginal NICHT(!) in den Papierkorb sondern in den Rei�wolf zu werfen.\\

Sollen mehrere Dateien in einem Container verschl�sselt werden, erstellt man ein Verzeichnis und kopiert die Dateien dort hinein. Anschlie�end verpackt man dieses Verzeichnis mit \textit{WinZip, 7zip} oder anderen Tools in ein Archiv und verschl�sselt dieses Archiv.\\

Wird die Option \textit{Symmetrisch verschl�sseln} gew�hlt, erfolgt die Verschl�sselung nicht mit einem Schl�ssel sondern nur mit einer Passphrase. Die Entschl�sselung erfordert dann ebenfalls nur die Angabe dieser Passphrase und keinen Key. Diese Variante wird f�r Backups empfohlen, die man auch nach einem Crash bei totalem Verlust aller Schl�ssel wieder herstellen will.\\

Zum Entschl�sseln reicht in der Regel ein Klick (oder Doppelklick) auf die verschl�sselte Datei. Nach Abfrage der Passphrase f�r den Schl�ssel liegt das entschl�sselte Orginal wieder auf der Platte.
