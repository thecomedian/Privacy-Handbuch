\subsubsection*{Content-Filter f�r JAP konfigurieren}
Im Anschlu� der Installation ist der Webrowser oder Content-Filter entsprechend zu konfigurieren, so dass Anfragen nicht direkt ins Internet geleitet werden, sondern �ber den JAP-Proxy. Wird ein Content-Filter (z.B. Privoxy oder Proxomitron) genutzt, ist die Kette \textit{Browser -> Filter -> JAP einzuhalten}.\\

\begin{itemize}
 \item \textbf{Privoxy:} F�r die Weiterleitung an den JAP Client ist folgende Zeile in der Datei \textit{config.txt} der Privoxy Konfiguration einzuf�gen. Linux-User finden die Konfiguration in der Regel im Verzeichnis /etc/privoxy (SuSe Linux: \textit{/var/lib/privoxy/etc}).
\begin{verbatim}
forward localhost:4001 .
\end{verbatim}
Wichtig: nicht den Punkt am Ende der Zeile vergessen!
\item \textbf{Proxomitron:} Ein Klick auf das Proxomitron-Symbol im Systray startet das Kontrollpanel von Proxomitron. Der Button \textit{Proxy} im Kontrollpanel �ffnet den im Bild \ref{abb:proxo_proxy} gezeigten Dialog. Eine neuen Proxy kann man mit einem Klick auf den Button \textit{Neu} hinzuf�gen. F�r den JAP Client ist in diesem Dialog \textit{localhost:4001} einzutragen.\\

\begin{figure}[htb]
\begin{center}
\includegraphics[scale=0.65]{../screenshots/proxo_settings_3.png}
\caption{Auswahl eines weiteren Proxy}
\label{abb:proxo_proxy}
\end{center}
\end{figure}

 Die Weiterleitung zur Mixkaskade kann mit der Option \textit{externen Proxy} nutzen im Kontrollpanel aktiviert und deaktiviert werden.
\end{itemize}
