\subsection{Linux konfigurieren}
Unter Linux sind nichts-standardm��ige Einstellungen leichter realiserbar. Es ist auch relativ einfach, einen lokalen DNS-Cache zu nutzen, um die zensurfreien DNS-Server nicht �berm��ig zu belasten.\\

\subsubsection*{pdnsd und resolvconf verwenden}
Der \textit{pdnsd} ist ein leichtgewichtiger DNS-Cache-Daemon. Er steht auf allen Linux-Distributionen zur Verf�gung. Unter Debian und Unbuntu installiert man ihn zusammen mit \textit{resolvconf}: 
\begin{verbatim}
 > sudo aptitude install resolvconf pdnsd 
\end{verbatim} 
Bei der Installation des \textit{pdnsd} wird man gefragt, wie die Namensaufl�sung erfolgen soll. W�hlen Sie zuerst einmal \textit{recursive}. Laden Sie die vorbereitete Konfigurationsdatei \href{https://www.awxcnx.de/download/pdnsd-gpfserver.conf}{https://www.awxcnx.de/download/pdnsd-gpfserver.conf} herunter und speichern Sie die Datei im Verzeichnis \textit{/usr/share/pdnsd}.\\ 

Anschlie�end in der Datei /etc/default/pdnsd den AUTO\_MODE anpassen:  
\begin{verbatim}
 START_DAEMON=yes
 AUTO_MODE=gpfserver
 OPTIONS=
\end{verbatim} 
Den Eigent�mer der Config-Datei auf \textit{root} setzen und den Daemon neu starten: 
\begin{verbatim}
 sudo chown root:root /usr/share/pdnsd/pdnsd-gpfserver.conf
 sudo invoke-rc.d pdnsd restart 
\end{verbatim} 
Der DNS-Traffic geht via TCP-Protokoll auf Port 110 zu den unzensierten DNS-Servern. Es ist schwer zu erkennen, dass es sich DNS-Traffic handelt und eine Umleitung auf DNS-Server der Provider ist wenig wahrscheinlich. Zur Sicherheit gelegentlich testen.

\subsubsection*{bind9 und resolvconf verwenden}
Die Pakete \textit{bind9} und \textit{resolvconf} sind in allen Distributionen fertig konfiguriert vorhanden und bieten einen vollst�ndigen DNS-Nameserver. Nach der Installation mit der Paketverwaltung l�uft der Nameserver und ist unter der Adresse 127.0.0.1 erreichbar. Die Tools aus dem Paket \textit{resolvconf} sorgen f�r die automatische Umkonfiguration, wenn \textit{bind9} gestartet und gestoppt wird. F�r Debian und Ubuntu k�nnen die Pakete mit \textit{aptitude} installiert werden:
\begin{verbatim}
 > sudo aptitude install resolvconf bind9
\end{verbatim} 
Die unzensierten DNS-Server sind in der Datei \textit{/etc/bind/named.conf.opions} einzutragen. Die Datei enth�lt bereits ein Muster. Dabei kann optional auch ein nicht �blicher Port angegeben werden:
\begin{verbatim}
 forwarders {
      94.75.228.29 port 110; 
      62.75.219.7  port 110;
 };
 listen-on { 127.0.0.1; };
\end{verbatim} 
(Standardm��ig lauscht der Daemon an allen Schnittstellen, auch an externen. Die Option \textit{listen-on} reduziert das auf den lokalen Rechner.)\\

Wer etwas ratlos ist, mit welchem Editor man eine Konfigurationsdatei anpasst, k�nnte \textit{``kdesu kwrite /etc/bind/named.conf.opions''} oder \textit{``gksu gedit /etc/bind/named.conf.opions''} probieren.\\

Nach der Anpassung der Konfiguration ist \textit{bind9} mitzuteilen, dass er die Konfigurationsdateien neu laden soll:
\begin{verbatim}
   > sudo invoke-rc.d bind9 reload
\end{verbatim} 
