\subsubsection*{Grafisches Interface f�r Linux}
Auch unter Linux steht das Tool \textit{Vidalia} als GUI f�r Tor zur Verf�gung. Das Programm ben�tigt Qt 4.1 oder h�her. Die Sourcen stehen auf der Downloadseite \href{http://tor.eff.org/dist/vidalia-bundles/}{http://tor.eff.org/dist/vidalia-bundles/} zur Verf�gung und werden mit dem �blichen Dreisatz installiert.\\

Eine Alternative ist das KDE-Programm \textbf{TorK}. Neben den Funktionen von Vidalia bietet es auch die M�glichkeit, viele Optionen einzustellen sowie diverse Programme f�r die Nutzung von Tor zu konfigurieren (Konqueror, Kopete, KSirc, SSH-Client). Es steht bereit unter \href{http://www.kde-apps.org/content/show.php?content=39442}{www.kde-apps.org/content/show.php?content=39442}\\

TorK ben�tigt f�r die �bersetzung das Entwickler-Packet \textit{kdebase-dev} sowie die davon abh�ngigen Packete. Unter Debian/GNU Linux wird au�erdem das Packet \textit{libkonq4-dev} ben�tigt. Nach dem Download ist das Archiv zu entpacken und als User root mit dem �blichen Dreisatz zu installieren:

\begin{verbatim}
   # configure --prefix=`kde-config --prefix`
   # make && make install
\end{verbatim}

Das Kommando \textit{kde-config --prefix} zur Ermittlung des KDE-Verzeichnisses ist in Backticks einzuschlie�en.\\

\begin{figure}[hbt]
\begin{center}
\includegraphics[scale=0.6]{../screenshots/tork_first_3.png}
\caption{TorK First Run Wizzard}
\label{abb:tork_first}
\end{center}
\end{figure}

Die Installationsroutine richtet einen Men�punkt in der Programmgruppe \textit{Internet} ein. Nach dem ersten Start f�hrt ein Wizzard in drei Schritten durch eine kurze Konfiguration. In den ersten beiden Schritten werden die Programme Tor und Privoxy abgefragt und in der Regel korrekt erkannt. Im dritten Schritt (Bild \ref{abb:tork_first}) wird das Verhalten von Tor festgelegt. Um die M�glichkeiten der grafischen Konfiguration zu nutzen, sollte in der Drop-Down-Box der Punkt \textit{Let me configure Tor} gew�hlt werden.\\

Im Anschluss k�nnen einige Einstellungen f�r Tor in der Konfiguration angepasst werden. In der Sektion \textit{My Client} kann ganz unten die Tor-Server Funktionalit�t abgeschaltet werden. Diese Option ist zu empfehlen, wenn der eigene Rechner hinter einem NAT-Gateway oder Firewall nicht vom Internet erreichbar ist.\\

\begin{figure}[hbt]
\begin{center}
\includegraphics[scale=0.7]{../screenshots/tork_config_1.png}
\caption{TorK Konfiguration}
\label{abb:tork_config}
\end{center}
\end{figure}

In der Sektion \textit{Konqueror} k�nnen die Einstellungen zur Sicherheit angepasst werden. Tork kann diese Optionen mit einem Knopfdruck in der Toolbar zusammen mit der Proxy-Umschaltung aktivieren.\\

Ein Klick auf die gr�ne Zwiebel in der Toolbar startet den Onion Router. Im Systray erscheint ebenfalls eine gr�ne Zwiebel. Ein Klick auf das Kreuz daneben beendet den Onion-Router. Wird TorK genutzt, sollte Tor immer vom GUI gestartet und beendet werden. Zwar erkennt TorK eine laufende Instanz des Onion-Routers, �ndert in diesem Fall aber sie Konfiguration selbst�ndig. Die Einstellungen m�ssen dann wieder neu zusammengestellt werden.\\

Nachdem der Onion Router zum ersten Mal 5-10 min gelaufen ist und �ber einen aktuellen Status des Netzwerkes verf�gt, kann man ihn wieder beenden und in den Einstellungen in der Sektion \textit{My Network View} bevorzugte Server als Exit-Knoten festlegen. W�hlt man hier schnelle Server aus Mitteleuropa, kann die Geschwindigkeit verbessert werden.\\

TorK bietet auch die M�glichkeit, mehrere Anwendungen f�r die Nutzung des Onion Routers zu konfigurieren oder so zu starten, dass die Daten �ber das Tor-Netzwerk geleitet werden. Dieses Feature findet man unter \textit{Anonymize}.\\

Die Option \textit{Anonymous Konqueror} �ndert die Proxy-Einstellungen im Kontrollzentrum und die Einstellungen zur Sicherheit (Java und JavaScript deaktivieren, Cache abschalten usw). Sie wirkt auf alle KDE-Programme, die diese Einstellungen nutzen (Quanta, Akregator usw.).\\

Leider werden beim Abschalten von TorK die Default-Einstellungen von Konqueror f�r Java, Javascript, Cache, Browserkennung und Proxy wieder hergestellt. Man muss nach der Nutzung von TOR diese Einstellungen f�r spurenarmes Surfen immer wieder anpassen oder einen anderen Browser f�r das Surfen ohne TOR nutzen.
