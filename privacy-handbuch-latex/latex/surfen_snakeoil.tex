\section{Snakeoil f�r Firefox (�berfl�ssiges)}
Auf der Mozilla-Website f�r Add-ons findet man tausende von Erweiterungen. Man kann nicht alle vorstellen. Ich bekomme immer wieder Hinweise auf dieses oder jenes privacyfreundliche Add-on und habe ein paar Dinge zusammengestellt, die ich nicht in die Empfehlungen aufnehme.\\

Als Grundsicherung empfehle ich die Kombination von \textit{CookieMonster} + \textit{NoScript} + \textit{AdBlock Plus} + \textit{HTTPS   Everywhere} + \textit{RefControl}. Viele Add-ons bieten Funktionen, die von dieser Kombination bereits abgedeckt werden. Andere sind einfach nur �berfl�ssig. 

\subsubsection*{Google Analytics Opt-Out}
Das Add-on von Google verhindert die Ausf�hrung der zu Google-Analytics geh�renden Scripte. Die Scripte werden jedoch trotzdem von den Google Servern geladen und man hinterl�sst Spuren in den Logdaten. Google erh�lt die Informationen zur IP-Adresse des Surfers und welche Webseite er gerade besucht (via Referer). Au�erdem gibt es �ber hundert weitere Surftracker, die ignoriert werden.\\

Die Add-ons NoScript und AdBlock erledigen diese Aufgabe besser.
Kategorie: \textit{echtes Snakeoil}

\subsubsection*{GoogleSharing}
Das Add-on verteilt alle Anfragen an die Google-Suche, Google-Cookies usw. �ber zentrale Server an zuf�llig ausgew�hlte Nutzer von GoogleSharing. Die Ergebnisse werden von den zuf�llig ausgew�hlten Nutzern �ber die zentralen Server zur�ck an den lokalen Firefox geliefert.\\

Nach unserer Meinung verbessert man seine Privatsph�re nicht, indem die Daten einem weiteren Dienst zur Verf�gung stellt. Das der eigene Rechner dabei auch unkontrolliert Daten von anderen Nutzern stellvertretend an Google weiterleitet, ist ein unn�tiges Risiko. Google speichert diese Informationen und gibt sie breitwillig an Beh�rden und Geheimdienste weiter. So kann man unschuldig in Verwicklungen geraten, die amn lieber vermeiden m�chte. Bei daten-speicherung.de findet man aktuelle Zahlen zur Datenweitergabe von Google an Beh�rden und Geheimdienste: 
\begin{itemize}
 \item 3x t�glich an deutsche Stellen
 \item 20x t�glich an US-amerikanische Stellen
 \item 6x t�glich an britische Stellen
\end{itemize}

Statt GoogleSharing sollte man lieber privacy-freundliche Alternativen nutzen: die Suchmaschine Ixquick.com oder Startingpage.com, f�r E-Mails einen Provider nutzen, der den Inhalt der Nachrichten nicht indexiert, openstreetmap.org statt Google-Maps verwenden\dots
Kategorie: \textit{gef�hrliches Snakeoil} 

\subsubsection*{Zweite Verteidigungslinie?}
Eine Reihe von Add-ons bieten Funktionen, welche durch die oben genannte Kombination bereits abgedeckt werden: 
\begin{itemize}
 \item \textit{FlashBlock} blockiert Flash-Animationen. Das erledigt auch NoScript.
 \item \textit{ForceHTTPS} kann f�r bestimmte Webseiten die Nutzung von HTTPS erzwingen, auch diese Funktion bietet NoScript.
 \item \textit{Targeted Advertising Cookie Opt-Out} und \textit{Ghostery} blockieren Surftracker. Es werden Trackingdienste blockiert, die auch AdBlock Plus mit der \textit{EasyPrivacy Liste} sehr gut blockiert. Au�erdem gibt es immer wieder Probleme mit \textit{Ghostery} auf einigen Webseiten, da das Add-on kein Whitelisting kennt.
 \item \textit{No FB Tracking} blockiert die Facebook Like Buttons. Auch das kann AdBlock Plus besser. Die SocialMediaBlock Liste von MontzA blockieren nicht nur Facebook Like Buttons sondern andere Social Networks.
 \item 
\end{itemize}

Wer meint, es nutzen zu m�ssen - Ok.